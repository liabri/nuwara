\documentclass{article}

\usepackage{booktabs}
\usepackage[french]{babel}
\usepackage{setspace}
\usepackage[T1]{fontenc}
\usepackage[a4paper, total={6in, 8in}]{geometry}
\usepackage{fontspec}
\usepackage{hyperref}

% fonts
\defaultfontfeatures{Mapping=tex-text,Scale=MatchLowercase}
\setmainfont{EB Garamond}

% variables
\newcommand\name{Maku }

% environments

\title{The \name Language}
\date{2021/11/05}
\author{Liam Brincat}

\begin{document}
    \maketitle
    \newpage

    \doublespacing
    \tableofcontents
    \doublespacing
    \newpage
    \section{Phonology}
    

    \subsection{Consonants}
        \name has 17 consonants with a three-way contrast between unvoiced segments, distinguished as plain, tense and aspirated.  

        \begin{center}
            \begin{tabular}{|c||c|c|c|c|c|c|c|}
            \hline
            & Labial & Alveolar & Palatal & Velar & Uvular & Glottal \\ 
            \hline
                Nasal & \textbf{\Large{m}} & \textbf{\Large{n}} & & & &\\  
            \hline
                Plosive & \textbf{\Large{p b}} & \textbf{\Large{t d}} & & \textbf{\Large{k g}} & & \\
            \hline
                Fricative & \textbf{\Large{f}} & \textbf{\Large{s}} & \textbf{\Large{ɕ ʑ}} & & &\\
            \hline
                Approximate & \textbf{\Large{w}} & \textbf{\Large{l}} & \textbf{\Large{j}} & & & \\
            \hline
                Rhotic tap/trill & & \textbf{\Large{r}} & & & & \\
            \hline
        \end{tabular}
        \end{center}

    \subsubsection{Gemination}
    \subsubsection{Positional Allophones}
    \subsection{Vowels}
    

          \begin{table}[htp]
     \centering
               \begin{tabular}{c|cccc}
                   \toprule
                   & Front & & Central & Back \\ 
                   & unrounded & rounded & & \\\midrule
                   Close & i & y & & u \\
                    Close-Mid & (e) & & & (o) \\          
                    Open-Mid & ɛ & & & ɔ \\                     
                    Open & & & a & \\ 
                   \bottomrule  
               \end{tabular}
        \caption{Vowels} 
          \end{table} 
 

    \subsubsection{Dipthongs and Glides}
    \subsection{Syllables}
    Standard \name may have the maximal form (C)(G)V(N/L)
    \subsection{Stress, rhythm and intonation}
    \subsection{Tones}
    \subsubsection{Sandhi}
   
    \newpage
    \section{Orthography}
    The modern makunian writing system uses an combination of logographic Dore, which are adapted from the historical and traditional writing system, and the modern alphasyllabic \emph{Mada}. The vast majority of sentences will require both scripts to be used in unisence, requiring deep knowledge in both to be able to write makunian.

    \subsection{Dore}
    \subsection{Mada}
    \subsubsection{Diacritics}
        1. Stress marks
        2. Voicing marks
        3. Palatalisation Marks ?

    \subsection{Direction}
    Traditionally, Makunian is written in a format called \emph{Naro}, which was like inheritated through \emph{Dore}. In \emph{Naro}, text is written top-down in vertical columns ordered from left to right. Contemporarly, Makunian is written in a format called \emph{Madokari} (Makunian: left-right), where the text is written horizontally from the left to the right. 

    The two formats are still used today, with the latter being used in certain literary pieces and such.
    
    \newpage
    \section{Grammar}
    Classification of Words 

    The modern standard of classification and one taught in education is known as the 9 Gamanjal system, dividing words into nine categories known as Gamanja.
    \subsection{Word Order}
    Standard word order is Subject-Object-Verb, being a fairly synthetic language allows the shifting of this order for emphasis. The emphasis will be on whatever is found at the end, for example the emphasis on the object is found in a SVO order.
    \subsection{Pronouns}

    \subsubsection{Personal}

    Maku pronouns are inflected based on person, number, gender and case. Consisting of all three persons, in both singular and plural senses. 

    Emphatic
    Person
    Inclusive
    Exclusive
    Direct
    Indirect
    Disjunctive

         \begin{table}[htp]
    \centering
              \begin{tabular}{c|c|ccc}
                  \toprule
                  Number & Person & Subject \\ \midrule
                  & 1$^{st}$ & mora \\
                  singular & 2$^{nd}$ & anoja \\
                  & 3$^{rd}$ & supin / supka \\ \midrule
                   & 1$^{st}$ & kimana \\          
                   plural & 2$^{nd}$ & dorja \\                     
                   & 3$^{rd}$ & zinra \\ 
                  \bottomrule  
              \end{tabular}
        \caption{Personal Pronouns} 
         \end{table} 


    \subsubsection{Demonstrative}

          \begin{table}[htp]
     \centering
               \begin{tabular}{c|cccc}
                   \toprule
                   & Proximal & Medial & Distal & Interrogative \\ \midrule
                   Place & mogi & dome & kampar & jomogi\\
                   Object & risa & garo & & jorisa \\
                   Time & ora & dop & & jora \\
                   Person & & & & jomin \\
                   Reason & & & & wo\\
                   Manner & & & & wera\\
                   One & darsi-mogi & darsi-dome & darsi-kampar & jol\\
                   From & &\\
                   To & &\\
                   \bottomrule  
               \end{tabular}
         \caption{Demonstrative Pronouns} 
          \end{table} 
 


    \subsection{Particles}
    \subsection{Numerals}
    \subsubsection{Cardinal}
    \subsubsection{Ordinal}
    \subsubsection{Counter Words}
    \subsection{Verbs}
    PRONOUNS: inflected par person, emphatic, inclusive/exclusive, direct/indirect, disjunctive


    Verbs are highly inflectional in a polysynthetic context
    Verbs have multiple suffixes which may be added together, notable valency, mood, tense/aspect and person.

    SCHWA IS COMMON AT THE END OF WORDS ?

    \subsubsection{Infinitive, -/y/ form}
    \subsubsection{Valency, -/o/ form}
    \subsubsection{Syntactic Mood}
        Imperative form is made by losing the final vowel.
    \subsubsection{Tense \& Aspect}
    \subsubsection{Person}

    There are multiple groups of verbs, 1, 2, 3.

    1. Within this group, a verb will change its final vowel when being conjugating to accord.
    
    2. The final vowel is constant throughout all conjugations
    
    3.  

    4. Irregular


    -a 
    -u
    -i
    -e
    -o 

    Person is conjugated in the prefix of the word


          \begin{table}[htp]
     \centering
               \begin{tabular}{c|c|ccc}
                   \toprule
                   Number & Person & Su \\ \midrule
                   & 1$^{st}$ & mora \\
                   singular & 2$^{nd}$ & anoja \\
                   & 3$^{rd}$ & supin / supka \\ \midrule
                    & 1$^{st}$ & kimana \\          
                    plural & 2$^{nd}$ & dorja \\                     
                    & 3$^{rd}$ & zinra \\ 
                   \bottomrule  
               \end{tabular}
        \caption{Personal Pronouns} 
          \end{table} 


    \subsubsection{Negation}

    Negation is shown by adding particle "ma" before and inflecting the suffix the verb by adding /z/.

    eg. ma \_x

    \subsection{Nouns}

    Noun classes:
        Animate
            Inflected by gender and number (singular, duel 
        Abstract
        Inanimate
            Inflected by possessive

    \subsection{Adjectives}
    \subsection{Adverbs}
    Pro drop language, but when used with emphasis may be added phrase-finally

    \newpage
    \section{Lexicon}


   \newpage 
   \listoffigures
\listoftables
\end{document}
